\documentclass[12pt]{scrartcl}

\usepackage[utf8]{inputenc}
\usepackage[naustrian]{babel}
\usepackage{caption}
\usepackage{graphicx}
\usepackage{verbatim}
\usepackage[T1]{fontenc}
\usepackage{lmodern}
\usepackage{subcaption}
\usepackage{amsmath}
\usepackage{amsfonts}
\usepackage{listings}
\usepackage{float}
\usepackage{tikz}
\usepackage{algpseudocode}
\usepackage{algorithm}
\usepackage{xcolor}

%pdfs
\usepackage{pdfpages}

%page borders
\usepackage{geometry}
\geometry{left=2.5cm,right=2.5cm,top=3cm,bottom=2.5cm}

\usepackage{minted}
\setminted {
	%style=igor, %borland, autumn, vs
	encoding=utf-8,
	autogobble,
	tabsize=4,
	linenos,
	breaklines,
	keywordcase=upper,
	%escapeinside=||
	%bgcolor=bg
	frame=single
}

\newenvironment{code}{\captionsetup{type=listing}}{}

%title/footer/header values
\usepackage{titling}
\title{AMS4 Cheetsheet}
\author{Elias Leonhardsberger}
\date{\today{}, Hagenberg}

%footer/header
%\usepackage[automark]{scrpage2}
%\pagestyle{headings}
%\clearscrheadfoot
%\ihead{\thetitle}
%\chead{\theauthor}
%\ohead{\today}
%\cfoot{Seite \pagemark}

\definecolor{aStarRed}{HTML}{fc6767}

\begin{document}
\section{Definitionen}
\begin{tabular}{ c | c | c }
    \textbf{Begriff}      & \textbf{Math. Def.}             & \textbf{Text Def.}                     \\
    \hline
    \textbf{Graph}        & $G = (V, E)$                    & Menge von Knoten $V$ und Kanten $E$    \\
    \hline
    \textbf{ungerichtet}  & $e = \{u, v\} \in E$            & Kanten verbinden Knoten ohne Richtung  \\
    \hline
    \textbf{gerichtet}    & $e = (u, v) \in E$              & Kanten verbinden Knoten mit Richtung   \\
    \hline
    \textbf{zusammen-}    & $G$ ist zusammenhängend         & Es gibt einen Pfad zwischen allen      \\
    \textbf{hängend}      &                                 & Knoten im Graphen $G$                  \\
    \hline
                          &                                 & Zwei Graphen sind isomorph, wenn es    \\
    \textbf{isomorph}     & $G_1 \cong G_2$                 & eine bijektive Abbildung zwischen      \\
                          &                                 & ihren Knoten gibt.                     \\
    \textbf{Brücke}       & $e \in E$ ist eine Brücke       & Entfernen von $e$ trennt den Graphen   \\
    \hline
    \textbf{Wanderung}    & $W = (v_1, v_2, \ldots, v_n)$   & Folge von Knoten,                      \\
                          &                                 & die durch Kanten verbunden sind        \\
    \hline
    \textbf{Weg}          & $P = (v_1, v_2, \ldots, v_n)$   & Wanderung ohne Kantenwiederholung      \\
    \hline
    \textbf{Pfad}         & $P = (v_1, v_2, \ldots, v_n)$   & Weg ohne Knotenwiederholung            \\
    \hline
    \textbf{Kreis}        & $C = (v_1, v_2, \ldots, v_n)$   & Weg, der am Startknoten endet          \\
                          &                                 &                                        \\
    \hline
    \textbf{Hamiltonpfad} & $P = (v_1, v_2, \ldots, v_n)$   & Pfad, der alle Knoten genau einmal     \\
                          &                                 & besucht                                \\
    \hline
    \textbf{Eulerpfad}    & $P = (v_1, v_2, \ldots, v_n)$   & Weg, der alle Kanten genau einmal      \\
                          &                                 & besucht                                \\
    \hline
    \textbf{Eulerkreis}   & $C = (v_1, v_2, \ldots, v_n)$   & Kreis, der alle Kanten genau einmal    \\
                          &                                 & besucht                                \\
    \hline
    \textbf{Handshaking}  & $\sum_{v \in V} \deg(v) = 2|E|$ & Summe der Grade aller Knoten ist       \\
    \textbf{Lemma}        &                                 & gleich der doppelten Anzahl der Kanten \\
    \hline
                          & $V = V_1 \cup V_2$              & Knotenmenge in zwei disjunkte Mengen   \\
    \textbf{bipartit}     & $E \subseteq V_1 \times V_2$    & Kanten verbinden nur Knoten aus        \\
                          &                                 & verschiedenen Mengen                   \\
    \hline
    \textbf{Wald}         & $G = (V, E)$                    & Graph ohne Zyklen                      \\
    \hline
    \textbf{Baum}         & $G = (V, E)$                    & Zusammenhängender, azyklischer Graph.  \\
                          & $|E| = |V| - 1$                 &                                        \\
    \hline
    \textbf{MST}          & $T = (V, E')$                   & Minimaler Spannbaum eines Graphen $G$  \\
                          &                                 & beinhaltet alle Knoten von $G$         \\
    \hline
    \textbf{Fluss-}       &                                 & Netzwerk $N$ mit Graph $G$             \\
    \textbf{netzwerk}     & $N = (G, w, s, t)$              & Kapazitätsfunktion $w$, Startknoten    \\
                          &                                 & $s$ und Zielknoten $t$                 \\
\end{tabular}
\pagebreak
\section{Algorithmen}
\begin{algorithm}
    \caption{Fleury}
    \begin{algorithmic}
        \Require Graph $G = (V, E)$
        \Ensure Graph ist zusammenhängend und hat entweder 0 oder 2 Knoten mit ungeradem Grad
        \State $C \gets \emptyset$ \Comment{Kreis}
        \State $v \gets getAnyNodeWithOddDegreeOrAnyNode(G)$ \Comment{Startknoten}
        \While{$E \neq \emptyset$}
        \State $e \gets getNeighboringNonBridge(v, E)$
        \If{$e = not found$}
        \State $e \gets getAnyNeighboringEdge(v, E)$
        \EndIf
        \If{$e = not found$}
        \State \Return C \Comment{Error: Kein Kreis gefunden}
        \EndIf
        \State $C \gets C \cup \{e\}$ \Comment{Füge Kante zum Kreis hinzu}
        \State $E \gets E \setminus \{e\}$ \Comment{Entferne Kante aus dem Graphen}
        \State $v \gets getOtherNode(e, v)$ \Comment{Wechsel zum anderen Knoten der Kante}
        \EndWhile
        \State \Return C \Comment{Gibt den Kreis zurück}
    \end{algorithmic}
\end{algorithm}

\pagebreak

\begin{algorithm}
    \caption{\fcolorbox{black}{yellow}{Bellman-Moore-Ford}, \fcolorbox{black}{cyan}{Dijkstra}, \fcolorbox{black}{aStarRed}{A*}}
    \begin{algorithmic}
        \Require Graph $G = (V, E)$, ein Startknoten $s$, ein Zielknoten $t$
        \Ensure Graph ist gewichtet, zusammenhängend und hat keine negativen Zyklen
        \Require \fcolorbox{black}{aStarRed}{A* benötigt eine Heuristik für die Abschätzung der Distanz.}
        \Ensure \fcolorbox{black}{aStarRed}{Die Heuristik muss monoton und optimistisch sein.($h(v) \leq dist(v, t)$)}

        \State $dist(s) \gets 0$ \Comment{Abstand vom Startknoten}
        \State $W(s) \gets \{s\}$ \Comment{Weg vom Startknoten}
        \State \fcolorbox{black}{magenta}{$F \gets \emptyset$} \Comment{Nur in Dijkstra und A*}
        \ForAll{$v \in V \setminus \{s\}$}
        \State $dist(v) \gets \infty$
        \State $W(v) \gets \emptyset$
        \EndFor

        \State \fcolorbox{black}{yellow}{$push(stack, s)$} \Comment{Stack für Bellman-Moore-Ford}
        \While{\fcolorbox{black}{yellow}{$stack \neq \emptyset$}} \Comment{In Dijkstra und A* ist die Bedingung \fcolorbox{black}{magenta}{$t \notin F$}}
        \State \fcolorbox{black}{yellow}{$v^* \gets pop(stack)$} \Comment{Nur in Bellman-Moore-Ford}
        \State \fcolorbox{black}{cyan}{$v^* \gets \arg\min_{v \in V} dist(v)$} \Comment{In Dijkstra}
        \State \fcolorbox{black}{aStarRed}{$v^* \gets \arg\min_{v \in V} (dist(v) + h(v))$} \Comment{In A* mit Heuristik $h$}
            \State \fcolorbox{black}{magenta}{$F \gets F \cup \{v^*\}$} \Comment{Nur in Dijkstra und A*}
            \ForAll{$v \in N(v^*)$} \Comment{Gehe alle Nachbarn von $v^*$ durch}
            \If{$dist(v^*) + l(v^*, v) < dist(v)$}
            \State $dist(v) \gets dist(v^*) + l(v^*, v)$
            \State $W(v) \gets W(v^*) \cup \{v\}$
        \State \fcolorbox{black}{yellow}{$push(stack, v)$} \Comment{Nur in Bellman-Moore-Ford}
        \EndIf
        \EndFor
        \EndWhile
        \State \Return dist, W \Comment{dist enthält alle Abstände, W alle Wege. Man kann auch nur den Abstand zum Zielknoten $t$ zurückgeben oder nur die Wege, je nach Nutzung.}
    \end{algorithmic}
\end{algorithm}
\pagebreak

\begin{algorithm}
    \caption{Prim}
    \begin{algorithmic}
        \Require Graph $G = (V, E)$, ein Wurzelknoten $r$
        \Ensure Graph ist zusammenhängend, ungerichtet und gewichtet
        \State $Q \gets V$
        \ForAll{$u \in Q$}
        \State $dist(u) \gets \infty$ \Comment{Abstand zur Wurzel}
        \State $pred(u) \gets 0$ \Comment{Vorgängerknoten im MST}
        \EndFor
        \State dist(r) $\gets 0$
        \While{$Q \neq \emptyset$}
        \State $u \gets \arg\min_{v \in Q} dist(v)$ \Comment{Knoten mit minimalem Abstand}
        \State $Q \gets Q \setminus \{u\}$
        \ForAll{$v \in N(u)$} \Comment{Gehe alle Nachbarn von $u$ durch}
        \If{$v \in Q \land l(u, v) < dist(v)$}
        \State $dist(v) \gets l(u, v)$
        \State $pred(v) \gets u$
        \EndIf
        \EndFor
        \EndWhile
    \end{algorithmic}
\end{algorithm}

\begin{algorithm}
    \caption{Kruskal}
    \begin{algorithmic}
        \Require Graph $G = (V, E)$
        \Ensure Graph ist zusammenhängend, ungerichtet und gewichtet
        \State $E' \gets \emptyset$ \Comment{MST Kanten}
        \State $L \gets sort(E)$ \Comment{Sortiere Kanten nach Gewicht}
        \While{$L \neq \emptyset$}
        \State $e \gets pop(L)$ \Comment{Nimm die leichteste Kante}
        \If{$(V, E' \cup \{e\})$ hat keinen Kreis}
        \State $E' \gets E' \cup \{e\}$ \Comment{Füge Kante zum MST hinzu}
        \EndIf
        \EndWhile
        \State \Return $(V, E')$ \Comment{MST}
    \end{algorithmic}
\end{algorithm}

\pagebreak

\begin{algorithm}
    \caption{Ford-Fulkerson}
    \begin{algorithmic}
        \Require Netzwerk $N = (G, w, s, t)$ mit Graph $G$, Kapazitätsfunktion $w$, Startknoten $s$ und Zielknoten $t$
        \Ensure $N$ ist ein Flussnetzwerk
        \State $f \gets 0$ \Comment{Aktueller Fluss}
        \While{es gibt einen augmentierenden Pfad $p$ von $s$ nach $t$ in $G_f$}
        \State $p \gets findAugmentingPath(s, t, G_f)$ \Comment{beliebiger augmentierenden Pfad}
        \State $c \gets \min_{e \in p} (w(e))$ \Comment{Bestimme die Flusskapazität}
        \State $f \gets f + c$
        \ForAll{$e \in p$}
        \State $w(e) \gets w(e) - c$
        \State $f(e_{rev}) \gets f(e_{rev}) + c$ \Comment{Füge den Fluss in die Rückwärtskante ein}
        \EndFor
        \EndWhile
        \State \Return f \Comment{Gibt den maximalen Fluss zurück}
    \end{algorithmic}
\end{algorithm}

\section{Ausgewählte Theoriefragen}
\pagebreak

\begin{itemize}
    \item \textbf{Welche Arten von Graphen kennen Sie?}
          \begin{itemize}
              \item Ungerichtete Graphen
              \item Gerichtete Graphen
              \item Gewichtete Graphen
              \item Zusammenhängende Graphen
              \item Azyklische Graphen (Wald)
              \item Azyklische zusammenhängende Graphen (Bäume)
              \item Bipartite Graphen
              \item Isomorphe Graphen
              \item Eulergraphen
              \item Hamiltongraphen
          \end{itemize}
    \item \textbf{Welche verschiedenen Darstellungsformen von Graphen kennen Sie?}
          \begin{itemize}
              \item Adjazenzmatrix => $A_{ij} = 1$ oder $A_{ij} = Gewicht$ wenn Kante zwischen $i$ und $j$ existiert, sonst $0$
              \item Adjazenzliste => Liste von Knoten, die jeweils ihre Nachbarn enthalten.
              \item Kantenliste => Liste aller Kanten, z.B. ${(u, v), (v, w)}$
          \end{itemize}
    \item \textbf{Wie wird das Kantengewicht mathematisch definiert?}
          \begin{itemize}
              \item Das Kantengewicht ist eine Funktion $l: E \to \mathbb{R}$, die jedem Kantenpaar $(u, v)$ ein Gewicht zuordnet.
              \item In gewichteten Graphen wird das Gewicht oft als Distanz oder Kosten interpretiert.
          \end{itemize}
    \item \textbf{Was ist Nachbarschaft in Graphen?}
          \begin{itemize}
              \item Die Nachbarschaft eines Knotens $v$ in einem Graphen $G = (V, E)$ ist die Menge aller Knoten, die direkt mit $v$ durch eine Kante verbunden sind.
              \item In gerichteten Graphen gibt es auch eingehende und ausgehende Nachbarn:
                    \begin{itemize}
                        \item Eingehende Nachbarn: $N_{in}(v) = \{u \in V | (u, v) \in E\}$
                        \item Ausgehende Nachbarn: $N_{out}(v) = \{u \in V | (v, u) \in E\}$
                        \item Nachbarn: $N(v) = N_{in}(v) \cup N_{out}(v)$
                    \end{itemize}
          \end{itemize}
    \item \textbf{Was ist ein regulärer Graph?}
          \begin{itemize}
              \item Ein regulärer Graph ist ein Graph, in dem alle Knoten den gleichen Grad haben.
              \item Formal: Ein Graph $G$ ist $k$-regulär, wenn $\deg(v) = k$ für alle $v \in V$ gilt.
              \item Beispiele:
                    \begin{itemize}
                        \item Ein 3-regulärer Graph hat für jeden Knoten genau 3 Nachbarn.
                        \item Ein vollständiger Graph $K_n$ ist $(n-1)$-regulär, da jeder Knoten mit allen anderen Knoten verbunden ist.
                    \end{itemize}
          \end{itemize}
          \pagebreak
    \item \textbf{Wann ist ein Graph eulersch?}
          \begin{itemize}
              \item Ein Graph ist eulersch, wenn er einen Eulerpfad enthält, der alle Kanten genau einmal besucht.
              \item Ein Graph ist eulersch, wenn er entweder:
                    \begin{itemize}
                        \item keinen Knoten mit ungeradem Grad hat (Eulerkreis) oder
                        \item genau zwei Knoten mit ungeradem Grad hat (Eulerpfad).
                    \end{itemize}
          \end{itemize}
    \item \textbf{Wann ist ein Graph hamiltonisch?}
          \begin{itemize}
              \item Ein Graph ist hamiltonisch, wenn er einen Hamiltonpfad enthält, der alle Knoten genau einmal besucht.
              \item Ein Graph ist hamiltonisch, wenn er einen Hamiltonkreis enthält, der alle Knoten genau einmal besucht und am Startknoten endet.
          \end{itemize}
    \item \textbf{Gegeben ein beliebiger Graph G, ist es hier möglich einen Eulerkreis zu konstruieren?}
          \begin{itemize}
              \item Nein, nicht immer.
              \item Ein Eulerkreis benötigt folgende Bedingungen:
                    \begin{itemize}
                        \item Der Graph muss zusammenhängend sein.
                        \item Alle Knoten müssen einen geraden Grad haben.
                    \end{itemize}
          \end{itemize}
    \item \textbf{Welche Algorithmen kennen Sie zur Berechnung von Eulerkreisen? Was sind die Unterschiede?}
          \begin{itemize}
              \item Fleury's Algorithmus:
                    \begin{itemize}
                        \item Geht Kanten durch und vermeidet Brücken, wenn möglich.
                        \item Einfach zu implementieren, aber ineffizient für große Graphen.
                    \end{itemize}
              \item Hierholzer's Algorithmus:
                    \begin{itemize}
                        \item Baut den Eulerkreis rekursiv auf.
                        \item Effizienter und schneller als Fleury's Algorithmus.
                    \end{itemize}
          \end{itemize}
    \item \textbf{Diskutieren Sie die Euler- bzw. Hamiltoneigenschaft für vollständige Graphen $K_n$ mit $n > 3$!}
          \begin{itemize}
              \item Vollständige Graphen $K_n$ sind hamiltonisch und manchmal eulerisch für $n > 3$.
              \item Euler: Jeder Knoten hat einen geraden Grad, falls $n$ ungerade ist(Kanten pro Knoten ist $n-1$).
                    Wenn $n$ gerade ist, hat jeder Knoten einen ungeraden Grad und es gibt keinen Eulerkreis.
              \item Hamilton: Jeder Knoten ist mit jedem anderen Knoten verbunden, was einen Hamiltonkreis ermöglicht.
          \end{itemize}
          \pagebreak
    \item \textbf{Diskutieren Sie die Lösbarkeit des Eulerkreisproblems mit dem des Hamiltonkreisproblems!}
          \begin{itemize}
              \item Das Eulerkreisproblem ist in polynomialer Zeit lösbar, während das Hamiltonkreisproblem NP-vollständig ist.
              \item Es gibt effiziente Algorithmen für spezielle Fälle des Eulerkreisproblems, während das Hamiltonkreisproblem in der Regel heuristische Ansätze erfordert.
          \end{itemize}
    \item \textbf{Wie lautet die Abbruchbedingung der bidirektionalen Suche?}
          \begin{itemize}
              \item Die na"ive Abbruchbedingung der bidirektionalen Suche ist erreicht, wenn die Suchfronten von Start- und Zielknoten sich treffen.
              \item Das bedeutet, dass ein Pfad zwischen den beiden Knoten gefunden wurde.
              \item
          \end{itemize}
    \item \textbf{In einem MST ist jede Verbindung die kürzeste Verbindung.}
          \begin{itemize}
              \item Falsch.
              \item Ein MST minimiert die Summe der Kantengewichte, aber nicht unbedingt die einzelnen Punkt-zu-Punkt-Verbindungen.
          \end{itemize}
    \item \textbf{Ist der Spannbaum aus dem Dijkstra Algorithmus minimal?}
          \begin{itemize}
              \item Falsch.
              \item Der Dijkstra-Algorithmus findet den kürzesten Pfad von einem Startknoten zu allen anderen Knoten, aber der resultierende Baum ist nicht unbedingt minimal im Sinne des Spannbaums.
          \end{itemize}
    \item \textbf{Welche Algorithmen für die Berechnung von Spannbäumen kennen Sie? Diskutieren Sie die Unterschiede!}
          \begin{itemize}
              \item Prim's Algorithmus:
                    \begin{itemize}
                        \item Startet bei einem Knoten und fügt iterativ die leichteste Kante hinzu, die einen neuen Knoten verbindet.
                        \item Effizient für dichte Graphen.
                    \end{itemize}
              \item Kruskal's Algorithmus:
                    \begin{itemize}
                        \item Sortiert die Kanten nach Gewicht und fügt sie zum MST hinzu, solange sie keinen Kreis bilden.
                        \item Effizient für spärliche Graphen.
                    \end{itemize}
          \end{itemize}
    \item \textbf{Wann ist ein MST eindeutig?}
          \begin{itemize}
              \item Ein MST ist eindeutig, wenn alle Kanten unterschiedliche Gewichte haben.
              \item Wenn zwei Kanten das gleiche Gewicht haben, kann es mehrere mögliche MSTs geben.
          \end{itemize}
          \pagebreak
    \item \textbf{Was unterscheidet eine Arboreszenz von einem gerichteten zyklenfreien Graphen?}
          \begin{itemize}
              \item Eine Arboreszenz ist ein gerichteter, zyklenfreier Graph, der von einem Wurzelknoten ausgeht und alle anderen Knoten erreicht.
              \item Ein gerichteter zyklenfreier Graph (DAG) kann mehrere Wurzelknoten haben und muss nicht notwendigerweise alle Knoten erreichen.
          \end{itemize}
    \item \textbf{Mit welchem Algorithmus findet man minimale Arboreszenzen?}
          \begin{itemize}
              \item Der Edmonds-Karp-Algorithmus ist ein bekannter Algorithmus zur Berechnung minimaler Arboreszenzen.
          \end{itemize}
    \item \textbf{Was ist ein Webgraph?}
          \begin{itemize}
              \item Ein Webgraph ist ein gerichteter Graph, der die Struktur des Internets oder eines Netzwerks von Webseiten darstellt.
              \item Knoten repräsentieren Webseiten, und Kanten repräsentieren Hyperlinks zwischen diesen Webseiten.
          \end{itemize}
    \item \textbf{Erklären Sie das Random Surfer Modell, welche Annahmen sind in dem Modell enthalten?}
          \begin{itemize}
              \item Das Random Surfer Modell beschreibt das Verhalten von Nutzern im Internet, die zufällig Links zwischen Webseiten folgen.
              \item Der Page Rank beschreibt die Chance, dass der Random Surfer auf einer bestimmten Seite landet.
          \end{itemize}
    \item \textbf{Was bedeutet der Dämpfungsfaktor im Random Surfer Modell?}
          \begin{itemize}
              \item Der Dämpfungsfaktor ist ein Wert zwischen 0 und 1, der die Wahrscheinlichkeit angibt, dass der Random Surfer einem Link folgt.
              \item Ein Dämpfungsfaktor von 0,85 bedeutet beispielsweise, dass 85\% der Zeit einem Link gefolgt wird und 15\% der Zeit eine zufällige Seite besucht wird.
              \item Dies verhindert, dass der Random Surfer in endlosen Schleifen stecken bleibt und sorgt für eine gleichmäßige Verteilung der Seitenbesuche.
          \end{itemize}
    \item \textbf{Welches Verfahren wird für die Berechnung des Page Rank verwendet?}
          \begin{itemize}
              \item Dafür wird das Random Surfer Modell verwendet, welches die Wahrscheinlichkeit beschreibt, dass ein Nutzer auf einer bestimmten Seite landet.
              \item Der Page Rank wird iterativ berechnet, indem die Adjazenzmatrix des Webgraphen verwendet wird.
              \item Der Dämpfungsfaktor wird dabei berücksichtigt, um die Wahrscheinlichkeit zu steuern, dass der Random Surfer einem Link folgt oder eine zufällige Seite besucht.
              \item Die Berechnung erfolgt durch die Lösung des Eigenwertproblems der Adjazenzmatrix, wobei der Page Rank als Eigenvektor des Matrixprodukts interpretiert wird.
          \end{itemize}
          \pagebreak
    \item \textbf{Ist eine Matrix irreduzibel und nicht-negativ, so gibt es auch einen positiven Eigenvektor.}
          \begin{itemize}
              \item Wahr.
              \item Eine irreduzible Matrix hat einen positiven Eigenvektor, da sie eine starke Verbindung zwischen den Zuständen darstellt.
          \end{itemize}
    \item \textbf{Beschreiben Sie ein weiteres Zentralitätsmaß außer die Eigenvektorzentralität für Graphen!}
          \begin{itemize}
              \item \textbf{Page Rank:} Misst die Wichtigkeit eines Knotens basierend auf der Anzahl und Qualität der eingehenden Verbindungen.
              \item \textbf{Zwischenzentralität:} Misst, wie oft ein Knoten auf dem kürzesten Pfad zwischen anderen Knoten liegt.
              \item \textbf{Nähezentralität:} Misst die durchschnittliche Entfernung eines Knotens zu allen anderen Knoten im Graphen.
          \end{itemize}
    \item \textbf{Was ist ein Netzwerk und was ist ein Fluss?}
          \begin{itemize}
              \item Ein Netzwerk ist ein Graph, wobei die Kanten Kapazitäten haben, die den maximalen Fluss zwischen den Knoten begrenzen.
              \item Ein Fluss nutzt die Kapazität aus, muss aber die Flusserhaltung und die Kapazitätsbeschränkungen der Kanten respektieren.
              \item Das Flusserhaltungsgesetz besagt, dass die Summe der eingehenden Flüsse gleich der Summe der ausgehenden Flüsse an jedem Knoten ist, außer am Start- und Zielknoten.
              \item $f(u) = \sum_{v \in N_{in}(u)} f(v) - \sum_{v \in N_{out}(u)} f(v)$, wobei $f(u)$ der Fluss am Knoten $u$ ist.
          \end{itemize}
    \item \textbf{Was ist ein Flussnetzwerk?}
          \begin{itemize}
              \item Ein Flussnetzwerk ist ein Netzwerk, in dessen Graphen es keine Retourkanten gibt, d.h. es gibt keine Kanten, die von einem Zielknoten zurück zu einem Startknoten führen.
          \end{itemize}
    \item \textbf{Wie lässt sich ein augmentierender Pfad finden?}
          \begin{itemize}
              \item Ein augmentierender Pfad ist ein Pfad im Flussnetzwerk, der von einem Startknoten zu einem Zielknoten führt und in dem noch Kapazität für zusätzlichen Fluss vorhanden ist.
              \item Er kann durch Tiefensuche (DFS) oder Breitensuche (BFS) gefunden werden.
          \end{itemize}
    \item \textbf{Lineare Programmierung ist nur dann effizient lösbar wenn alle Variablen kontinuierlich sind.}
          \begin{itemize}
              \item Falsch.
              \item Lineare Programmierung kann auch mit ganzzahligen Variablen gelöst werden, aber die Effizienz kann variieren.
          \end{itemize}
\end{itemize}
\end{document}